% Options for packages loaded elsewhere
\PassOptionsToPackage{unicode}{hyperref}
\PassOptionsToPackage{hyphens}{url}
\PassOptionsToPackage{dvipsnames,svgnames,x11names}{xcolor}
%
\documentclass[
  letterpaper,
  DIV=11,
  numbers=noendperiod]{scrartcl}

\usepackage{amsmath,amssymb}
\usepackage{iftex}
\ifPDFTeX
  \usepackage[T1]{fontenc}
  \usepackage[utf8]{inputenc}
  \usepackage{textcomp} % provide euro and other symbols
\else % if luatex or xetex
  \usepackage{unicode-math}
  \defaultfontfeatures{Scale=MatchLowercase}
  \defaultfontfeatures[\rmfamily]{Ligatures=TeX,Scale=1}
\fi
\usepackage{lmodern}
\ifPDFTeX\else  
    % xetex/luatex font selection
\fi
% Use upquote if available, for straight quotes in verbatim environments
\IfFileExists{upquote.sty}{\usepackage{upquote}}{}
\IfFileExists{microtype.sty}{% use microtype if available
  \usepackage[]{microtype}
  \UseMicrotypeSet[protrusion]{basicmath} % disable protrusion for tt fonts
}{}
\makeatletter
\@ifundefined{KOMAClassName}{% if non-KOMA class
  \IfFileExists{parskip.sty}{%
    \usepackage{parskip}
  }{% else
    \setlength{\parindent}{0pt}
    \setlength{\parskip}{6pt plus 2pt minus 1pt}}
}{% if KOMA class
  \KOMAoptions{parskip=half}}
\makeatother
\usepackage{xcolor}
\setlength{\emergencystretch}{3em} % prevent overfull lines
\setcounter{secnumdepth}{-\maxdimen} % remove section numbering
% Make \paragraph and \subparagraph free-standing
\ifx\paragraph\undefined\else
  \let\oldparagraph\paragraph
  \renewcommand{\paragraph}[1]{\oldparagraph{#1}\mbox{}}
\fi
\ifx\subparagraph\undefined\else
  \let\oldsubparagraph\subparagraph
  \renewcommand{\subparagraph}[1]{\oldsubparagraph{#1}\mbox{}}
\fi

\usepackage{color}
\usepackage{fancyvrb}
\newcommand{\VerbBar}{|}
\newcommand{\VERB}{\Verb[commandchars=\\\{\}]}
\DefineVerbatimEnvironment{Highlighting}{Verbatim}{commandchars=\\\{\}}
% Add ',fontsize=\small' for more characters per line
\usepackage{framed}
\definecolor{shadecolor}{RGB}{241,243,245}
\newenvironment{Shaded}{\begin{snugshade}}{\end{snugshade}}
\newcommand{\AlertTok}[1]{\textcolor[rgb]{0.68,0.00,0.00}{#1}}
\newcommand{\AnnotationTok}[1]{\textcolor[rgb]{0.37,0.37,0.37}{#1}}
\newcommand{\AttributeTok}[1]{\textcolor[rgb]{0.40,0.45,0.13}{#1}}
\newcommand{\BaseNTok}[1]{\textcolor[rgb]{0.68,0.00,0.00}{#1}}
\newcommand{\BuiltInTok}[1]{\textcolor[rgb]{0.00,0.23,0.31}{#1}}
\newcommand{\CharTok}[1]{\textcolor[rgb]{0.13,0.47,0.30}{#1}}
\newcommand{\CommentTok}[1]{\textcolor[rgb]{0.37,0.37,0.37}{#1}}
\newcommand{\CommentVarTok}[1]{\textcolor[rgb]{0.37,0.37,0.37}{\textit{#1}}}
\newcommand{\ConstantTok}[1]{\textcolor[rgb]{0.56,0.35,0.01}{#1}}
\newcommand{\ControlFlowTok}[1]{\textcolor[rgb]{0.00,0.23,0.31}{#1}}
\newcommand{\DataTypeTok}[1]{\textcolor[rgb]{0.68,0.00,0.00}{#1}}
\newcommand{\DecValTok}[1]{\textcolor[rgb]{0.68,0.00,0.00}{#1}}
\newcommand{\DocumentationTok}[1]{\textcolor[rgb]{0.37,0.37,0.37}{\textit{#1}}}
\newcommand{\ErrorTok}[1]{\textcolor[rgb]{0.68,0.00,0.00}{#1}}
\newcommand{\ExtensionTok}[1]{\textcolor[rgb]{0.00,0.23,0.31}{#1}}
\newcommand{\FloatTok}[1]{\textcolor[rgb]{0.68,0.00,0.00}{#1}}
\newcommand{\FunctionTok}[1]{\textcolor[rgb]{0.28,0.35,0.67}{#1}}
\newcommand{\ImportTok}[1]{\textcolor[rgb]{0.00,0.46,0.62}{#1}}
\newcommand{\InformationTok}[1]{\textcolor[rgb]{0.37,0.37,0.37}{#1}}
\newcommand{\KeywordTok}[1]{\textcolor[rgb]{0.00,0.23,0.31}{#1}}
\newcommand{\NormalTok}[1]{\textcolor[rgb]{0.00,0.23,0.31}{#1}}
\newcommand{\OperatorTok}[1]{\textcolor[rgb]{0.37,0.37,0.37}{#1}}
\newcommand{\OtherTok}[1]{\textcolor[rgb]{0.00,0.23,0.31}{#1}}
\newcommand{\PreprocessorTok}[1]{\textcolor[rgb]{0.68,0.00,0.00}{#1}}
\newcommand{\RegionMarkerTok}[1]{\textcolor[rgb]{0.00,0.23,0.31}{#1}}
\newcommand{\SpecialCharTok}[1]{\textcolor[rgb]{0.37,0.37,0.37}{#1}}
\newcommand{\SpecialStringTok}[1]{\textcolor[rgb]{0.13,0.47,0.30}{#1}}
\newcommand{\StringTok}[1]{\textcolor[rgb]{0.13,0.47,0.30}{#1}}
\newcommand{\VariableTok}[1]{\textcolor[rgb]{0.07,0.07,0.07}{#1}}
\newcommand{\VerbatimStringTok}[1]{\textcolor[rgb]{0.13,0.47,0.30}{#1}}
\newcommand{\WarningTok}[1]{\textcolor[rgb]{0.37,0.37,0.37}{\textit{#1}}}

\providecommand{\tightlist}{%
  \setlength{\itemsep}{0pt}\setlength{\parskip}{0pt}}\usepackage{longtable,booktabs,array}
\usepackage{calc} % for calculating minipage widths
% Correct order of tables after \paragraph or \subparagraph
\usepackage{etoolbox}
\makeatletter
\patchcmd\longtable{\par}{\if@noskipsec\mbox{}\fi\par}{}{}
\makeatother
% Allow footnotes in longtable head/foot
\IfFileExists{footnotehyper.sty}{\usepackage{footnotehyper}}{\usepackage{footnote}}
\makesavenoteenv{longtable}
\usepackage{graphicx}
\makeatletter
\def\maxwidth{\ifdim\Gin@nat@width>\linewidth\linewidth\else\Gin@nat@width\fi}
\def\maxheight{\ifdim\Gin@nat@height>\textheight\textheight\else\Gin@nat@height\fi}
\makeatother
% Scale images if necessary, so that they will not overflow the page
% margins by default, and it is still possible to overwrite the defaults
% using explicit options in \includegraphics[width, height, ...]{}
\setkeys{Gin}{width=\maxwidth,height=\maxheight,keepaspectratio}
% Set default figure placement to htbp
\makeatletter
\def\fps@figure{htbp}
\makeatother

\KOMAoption{captions}{tableheading}
\makeatletter
\makeatother
\makeatletter
\makeatother
\makeatletter
\@ifpackageloaded{caption}{}{\usepackage{caption}}
\AtBeginDocument{%
\ifdefined\contentsname
  \renewcommand*\contentsname{Table of contents}
\else
  \newcommand\contentsname{Table of contents}
\fi
\ifdefined\listfigurename
  \renewcommand*\listfigurename{List of Figures}
\else
  \newcommand\listfigurename{List of Figures}
\fi
\ifdefined\listtablename
  \renewcommand*\listtablename{List of Tables}
\else
  \newcommand\listtablename{List of Tables}
\fi
\ifdefined\figurename
  \renewcommand*\figurename{Figure}
\else
  \newcommand\figurename{Figure}
\fi
\ifdefined\tablename
  \renewcommand*\tablename{Table}
\else
  \newcommand\tablename{Table}
\fi
}
\@ifpackageloaded{float}{}{\usepackage{float}}
\floatstyle{ruled}
\@ifundefined{c@chapter}{\newfloat{codelisting}{h}{lop}}{\newfloat{codelisting}{h}{lop}[chapter]}
\floatname{codelisting}{Listing}
\newcommand*\listoflistings{\listof{codelisting}{List of Listings}}
\makeatother
\makeatletter
\@ifpackageloaded{caption}{}{\usepackage{caption}}
\@ifpackageloaded{subcaption}{}{\usepackage{subcaption}}
\makeatother
\makeatletter
\@ifpackageloaded{tcolorbox}{}{\usepackage[skins,breakable]{tcolorbox}}
\makeatother
\makeatletter
\@ifundefined{shadecolor}{\definecolor{shadecolor}{rgb}{.97, .97, .97}}
\makeatother
\makeatletter
\makeatother
\makeatletter
\makeatother
\ifLuaTeX
  \usepackage{selnolig}  % disable illegal ligatures
\fi
\IfFileExists{bookmark.sty}{\usepackage{bookmark}}{\usepackage{hyperref}}
\IfFileExists{xurl.sty}{\usepackage{xurl}}{} % add URL line breaks if available
\urlstyle{same} % disable monospaced font for URLs
\hypersetup{
  pdftitle={Guide de bonne pratique sur R},
  pdfauthor={Aveneau},
  colorlinks=true,
  linkcolor={blue},
  filecolor={Maroon},
  citecolor={Blue},
  urlcolor={Blue},
  pdfcreator={LaTeX via pandoc}}

\title{Guide de bonne pratique sur R}
\author{Aveneau}
\date{}

\begin{document}
\maketitle
\ifdefined\Shaded\renewenvironment{Shaded}{\begin{tcolorbox}[boxrule=0pt, borderline west={3pt}{0pt}{shadecolor}, interior hidden, sharp corners, breakable, enhanced, frame hidden]}{\end{tcolorbox}}\fi

\hypertarget{bons-conseils-pour-coder-sur-r}{%
\subsection{5 bons conseils pour coder sur
R}\label{bons-conseils-pour-coder-sur-r}}

Le but de ce document est d'avoir une trace des conseils donnés lors du
cours sur openclassrroom.

\hypertarget{conseil-n1-maintenir-les-packages-r-et-rstudio-uxe0-jour}{%
\subsection{\texorpdfstring{\textbf{Conseil n°1} : Maintenir les
packages, R et Rstudio à
jour}{Conseil n°1 : Maintenir les packages, R et Rstudio à jour}}\label{conseil-n1-maintenir-les-packages-r-et-rstudio-uxe0-jour}}

Les packages peuvent être mis à jour dans l'onglet \texttt{Packages}
puis en cliquant sur \texttt{Update} ou dans le menu
\texttt{Tools\ \textgreater{}\ Check\ for\ Package\ Update}. \textbf{A
faire au moins une fois par mois.}

Lorsqu'une nouvelle version de R est disponible, il est conseillé de la
télécharger dans un nouveau dossier à part (ce qui est fait par
défaut).\\
Il n'y a aucun problème à avoir plusieurs versions de R sur son
ordinateur, il faut par contre vérifier que la bonne est bien utilisée
(premier texte affiché dans la console ou \texttt{sessionInfo()} ou
\texttt{Tools\ \textgreater{}\ Global\ Options\ \textgreater{}\ R\ general\ \textgreater{}\ R\ version}).

Enfin, RStudio informe quand une nouvelle mise à jour est disponible.

\hypertarget{conseil-n2-ne-pas-enregistrer-les-donnuxe9es-dans-un-fichier-.rdata-uxe0-la-fermeture-dune-session}{%
\subsection{\texorpdfstring{\textbf{Conseil n°2} : Ne pas enregistrer
les données dans un fichier \texttt{.RData} à la fermeture d'une
session}{Conseil n°2 : Ne pas enregistrer les données dans un fichier .RData à la fermeture d'une session}}\label{conseil-n2-ne-pas-enregistrer-les-donnuxe9es-dans-un-fichier-.rdata-uxe0-la-fermeture-dune-session}}

Comme expliqué dans la session ``Ajustez l'apparence de RStudio'', je
vous conseille de ne pas enregistrer les données à la fermeture de la
session.\\
Mais si vous voulez que RStudio recharge les données en plus des
fichiers ouverts, il faut dans
\texttt{Tools\ \textgreater{}\ Global\ options\ \textgreater{}\ General\ \textgreater{}\ Basic\ \textgreater{}\ Workspace}
cocher \texttt{Restore\ .RData\ into\ workspace\ at\ startup}.\\
Vous pouvez aussi sélectionner \texttt{Always} pour
\texttt{Save\ workspace\ to\ .RData\ on\ exit:} si vous voulez que vos
données soient enregistrées et restaurées automatiquement.\\
Ces options sont déconseillées aux personnes qui débutent car vous
pouvez oublier les modifications déjà réalisées sur les données.\\
Imaginez que que pour votre peinture, vous deviez rajouter un flacon de
couleur dans le blanc, à la reprise des travaux vous pouvez vous
demander si vous l'avez déjà fait pour ce pot ou si vous devez le faire
maintenant.\\
Une bonne pratique est de relancer les scripts à chaque ouverture de
RStudio afin d'être sûre des modifications réalisées sur les données.

\hypertarget{conseil-n3-apprenez-uxe0-bien-guxe9rer-git-avec-rstudio}{%
\subsection{\texorpdfstring{\textbf{Conseil n°3} : Apprenez à bien gérer
\texttt{Git} avec
\texttt{RStudio}}{Conseil n°3 : Apprenez à bien gérer Git avec RStudio}}\label{conseil-n3-apprenez-uxe0-bien-guxe9rer-git-avec-rstudio}}

Pour apprendre à gérer parfaitement \texttt{Git} et \texttt{RStudio}, je
vous recommande de lire les ressources :

\_
\href{https://www.book.utilitr.org/03_fiches_thematiques/fiche_git_utilisation}{Utiliser
\texttt{Git} avec RStudio} du projet \texttt{\{utilitR\}}de
\emph{INSEE}\\
\_
\href{https://linogaliana.gitlab.io/collaboratif/git.html\#des-bases-de-git}{Utiliser
\texttt{GIT} avec R} de Lino GALIANA extrait de Travail collaboratif
avec R\\
\_ L'article
\href{https://thinkr.fr/travailler-avec-git-via-rstudio-et-versionner-son-code/}{Travailler
avec Git via RStudio et versionner son code} du blog de \emph{ThinkR}
par Elena SALETTE

\hypertarget{conseil-n4-enregistrez-ruxe9guliuxe8rement-vos-fichiers}{%
\subsection{\texorpdfstring{\textbf{Conseil n°4} : Enregistrez
régulièrement vos
fichiers}{Conseil n°4 : Enregistrez régulièrement vos fichiers}}\label{conseil-n4-enregistrez-ruxe9guliuxe8rement-vos-fichiers}}

Par défaut, le fichier doit-être enregistré pour être compilé donc vous
allez enregistrez au moins à chaque compilation avec \texttt{Render}
mais c'est mieux de le faire plus souvent.\\
Par rapport à \texttt{Git}, il est conseillé de faire un \texttt{commit}
(enregistrer une version d'un ou plusieurs fichier(s)) avant et après
chaque développement. Par exemple, faites un \texttt{commit} avant
d'importer vos données dans R et après avoir écrit le code et les avoir
importées.

\hypertarget{conseil-n5-attention-uxe0-la-casse}{%
\subsection{\texorpdfstring{\textbf{Conseil n°5} : Attention à la
casse}{Conseil n°5 : Attention à la casse}}\label{conseil-n5-attention-uxe0-la-casse}}

R comme beaucoup de langage de programmation est sensible à la casse,
c'est-à-dire qu'il fait la différence entre majuscule et minuscule. Si
vous créez l'objet \texttt{A} puis ensuite que vous écrivez
\texttt{a\ *\ 2}, R affiche une erreur en vous disant que \texttt{a} est
un objet inconnu.

\begin{Shaded}
\begin{Highlighting}[]
\CommentTok{\# assignation d\textquotesingle{}une valeur à l\textquotesingle{}objet a}
\NormalTok{a }\OtherTok{\textless{}{-}} \DecValTok{2}

\CommentTok{\# multiplier par deux}
\NormalTok{A }\SpecialCharTok{*} \DecValTok{2}
\end{Highlighting}
\end{Shaded}

\begin{verbatim}
Error in eval(expr, envir, enclos): objet 'A' introuvable
\end{verbatim}

\begin{Shaded}
\begin{Highlighting}[]
\CommentTok{\# attention à la casse}
\NormalTok{a }\SpecialCharTok{*} \DecValTok{2}
\end{Highlighting}
\end{Shaded}

\begin{verbatim}
[1] 4
\end{verbatim}



\end{document}
