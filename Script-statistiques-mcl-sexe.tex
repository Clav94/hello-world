% Options for packages loaded elsewhere
\PassOptionsToPackage{unicode}{hyperref}
\PassOptionsToPackage{hyphens}{url}
\PassOptionsToPackage{dvipsnames,svgnames,x11names}{xcolor}
%
\documentclass[
  letterpaper,
  DIV=11,
  numbers=noendperiod]{scrartcl}

\usepackage{amsmath,amssymb}
\usepackage{iftex}
\ifPDFTeX
  \usepackage[T1]{fontenc}
  \usepackage[utf8]{inputenc}
  \usepackage{textcomp} % provide euro and other symbols
\else % if luatex or xetex
  \usepackage{unicode-math}
  \defaultfontfeatures{Scale=MatchLowercase}
  \defaultfontfeatures[\rmfamily]{Ligatures=TeX,Scale=1}
\fi
\usepackage{lmodern}
\ifPDFTeX\else  
    % xetex/luatex font selection
\fi
% Use upquote if available, for straight quotes in verbatim environments
\IfFileExists{upquote.sty}{\usepackage{upquote}}{}
\IfFileExists{microtype.sty}{% use microtype if available
  \usepackage[]{microtype}
  \UseMicrotypeSet[protrusion]{basicmath} % disable protrusion for tt fonts
}{}
\makeatletter
\@ifundefined{KOMAClassName}{% if non-KOMA class
  \IfFileExists{parskip.sty}{%
    \usepackage{parskip}
  }{% else
    \setlength{\parindent}{0pt}
    \setlength{\parskip}{6pt plus 2pt minus 1pt}}
}{% if KOMA class
  \KOMAoptions{parskip=half}}
\makeatother
\usepackage{xcolor}
\setlength{\emergencystretch}{3em} % prevent overfull lines
\setcounter{secnumdepth}{-\maxdimen} % remove section numbering
% Make \paragraph and \subparagraph free-standing
\ifx\paragraph\undefined\else
  \let\oldparagraph\paragraph
  \renewcommand{\paragraph}[1]{\oldparagraph{#1}\mbox{}}
\fi
\ifx\subparagraph\undefined\else
  \let\oldsubparagraph\subparagraph
  \renewcommand{\subparagraph}[1]{\oldsubparagraph{#1}\mbox{}}
\fi

\usepackage{color}
\usepackage{fancyvrb}
\newcommand{\VerbBar}{|}
\newcommand{\VERB}{\Verb[commandchars=\\\{\}]}
\DefineVerbatimEnvironment{Highlighting}{Verbatim}{commandchars=\\\{\}}
% Add ',fontsize=\small' for more characters per line
\usepackage{framed}
\definecolor{shadecolor}{RGB}{241,243,245}
\newenvironment{Shaded}{\begin{snugshade}}{\end{snugshade}}
\newcommand{\AlertTok}[1]{\textcolor[rgb]{0.68,0.00,0.00}{#1}}
\newcommand{\AnnotationTok}[1]{\textcolor[rgb]{0.37,0.37,0.37}{#1}}
\newcommand{\AttributeTok}[1]{\textcolor[rgb]{0.40,0.45,0.13}{#1}}
\newcommand{\BaseNTok}[1]{\textcolor[rgb]{0.68,0.00,0.00}{#1}}
\newcommand{\BuiltInTok}[1]{\textcolor[rgb]{0.00,0.23,0.31}{#1}}
\newcommand{\CharTok}[1]{\textcolor[rgb]{0.13,0.47,0.30}{#1}}
\newcommand{\CommentTok}[1]{\textcolor[rgb]{0.37,0.37,0.37}{#1}}
\newcommand{\CommentVarTok}[1]{\textcolor[rgb]{0.37,0.37,0.37}{\textit{#1}}}
\newcommand{\ConstantTok}[1]{\textcolor[rgb]{0.56,0.35,0.01}{#1}}
\newcommand{\ControlFlowTok}[1]{\textcolor[rgb]{0.00,0.23,0.31}{#1}}
\newcommand{\DataTypeTok}[1]{\textcolor[rgb]{0.68,0.00,0.00}{#1}}
\newcommand{\DecValTok}[1]{\textcolor[rgb]{0.68,0.00,0.00}{#1}}
\newcommand{\DocumentationTok}[1]{\textcolor[rgb]{0.37,0.37,0.37}{\textit{#1}}}
\newcommand{\ErrorTok}[1]{\textcolor[rgb]{0.68,0.00,0.00}{#1}}
\newcommand{\ExtensionTok}[1]{\textcolor[rgb]{0.00,0.23,0.31}{#1}}
\newcommand{\FloatTok}[1]{\textcolor[rgb]{0.68,0.00,0.00}{#1}}
\newcommand{\FunctionTok}[1]{\textcolor[rgb]{0.28,0.35,0.67}{#1}}
\newcommand{\ImportTok}[1]{\textcolor[rgb]{0.00,0.46,0.62}{#1}}
\newcommand{\InformationTok}[1]{\textcolor[rgb]{0.37,0.37,0.37}{#1}}
\newcommand{\KeywordTok}[1]{\textcolor[rgb]{0.00,0.23,0.31}{#1}}
\newcommand{\NormalTok}[1]{\textcolor[rgb]{0.00,0.23,0.31}{#1}}
\newcommand{\OperatorTok}[1]{\textcolor[rgb]{0.37,0.37,0.37}{#1}}
\newcommand{\OtherTok}[1]{\textcolor[rgb]{0.00,0.23,0.31}{#1}}
\newcommand{\PreprocessorTok}[1]{\textcolor[rgb]{0.68,0.00,0.00}{#1}}
\newcommand{\RegionMarkerTok}[1]{\textcolor[rgb]{0.00,0.23,0.31}{#1}}
\newcommand{\SpecialCharTok}[1]{\textcolor[rgb]{0.37,0.37,0.37}{#1}}
\newcommand{\SpecialStringTok}[1]{\textcolor[rgb]{0.13,0.47,0.30}{#1}}
\newcommand{\StringTok}[1]{\textcolor[rgb]{0.13,0.47,0.30}{#1}}
\newcommand{\VariableTok}[1]{\textcolor[rgb]{0.07,0.07,0.07}{#1}}
\newcommand{\VerbatimStringTok}[1]{\textcolor[rgb]{0.13,0.47,0.30}{#1}}
\newcommand{\WarningTok}[1]{\textcolor[rgb]{0.37,0.37,0.37}{\textit{#1}}}

\providecommand{\tightlist}{%
  \setlength{\itemsep}{0pt}\setlength{\parskip}{0pt}}\usepackage{longtable,booktabs,array}
\usepackage{calc} % for calculating minipage widths
% Correct order of tables after \paragraph or \subparagraph
\usepackage{etoolbox}
\makeatletter
\patchcmd\longtable{\par}{\if@noskipsec\mbox{}\fi\par}{}{}
\makeatother
% Allow footnotes in longtable head/foot
\IfFileExists{footnotehyper.sty}{\usepackage{footnotehyper}}{\usepackage{footnote}}
\makesavenoteenv{longtable}
\usepackage{graphicx}
\makeatletter
\def\maxwidth{\ifdim\Gin@nat@width>\linewidth\linewidth\else\Gin@nat@width\fi}
\def\maxheight{\ifdim\Gin@nat@height>\textheight\textheight\else\Gin@nat@height\fi}
\makeatother
% Scale images if necessary, so that they will not overflow the page
% margins by default, and it is still possible to overwrite the defaults
% using explicit options in \includegraphics[width, height, ...]{}
\setkeys{Gin}{width=\maxwidth,height=\maxheight,keepaspectratio}
% Set default figure placement to htbp
\makeatletter
\def\fps@figure{htbp}
\makeatother

\KOMAoption{captions}{tableheading}
\makeatletter
\makeatother
\makeatletter
\makeatother
\makeatletter
\@ifpackageloaded{caption}{}{\usepackage{caption}}
\AtBeginDocument{%
\ifdefined\contentsname
  \renewcommand*\contentsname{Table of contents}
\else
  \newcommand\contentsname{Table of contents}
\fi
\ifdefined\listfigurename
  \renewcommand*\listfigurename{List of Figures}
\else
  \newcommand\listfigurename{List of Figures}
\fi
\ifdefined\listtablename
  \renewcommand*\listtablename{List of Tables}
\else
  \newcommand\listtablename{List of Tables}
\fi
\ifdefined\figurename
  \renewcommand*\figurename{Figure}
\else
  \newcommand\figurename{Figure}
\fi
\ifdefined\tablename
  \renewcommand*\tablename{Table}
\else
  \newcommand\tablename{Table}
\fi
}
\@ifpackageloaded{float}{}{\usepackage{float}}
\floatstyle{ruled}
\@ifundefined{c@chapter}{\newfloat{codelisting}{h}{lop}}{\newfloat{codelisting}{h}{lop}[chapter]}
\floatname{codelisting}{Listing}
\newcommand*\listoflistings{\listof{codelisting}{List of Listings}}
\makeatother
\makeatletter
\@ifpackageloaded{caption}{}{\usepackage{caption}}
\@ifpackageloaded{subcaption}{}{\usepackage{subcaption}}
\makeatother
\makeatletter
\@ifpackageloaded{tcolorbox}{}{\usepackage[skins,breakable]{tcolorbox}}
\makeatother
\makeatletter
\@ifundefined{shadecolor}{\definecolor{shadecolor}{rgb}{.97, .97, .97}}
\makeatother
\makeatletter
\makeatother
\makeatletter
\makeatother
\ifLuaTeX
  \usepackage{selnolig}  % disable illegal ligatures
\fi
\IfFileExists{bookmark.sty}{\usepackage{bookmark}}{\usepackage{hyperref}}
\IfFileExists{xurl.sty}{\usepackage{xurl}}{} % add URL line breaks if available
\urlstyle{same} % disable monospaced font for URLs
\hypersetup{
  pdftitle={Analyse statistique entre sexes},
  pdfauthor={Aveneau / lange},
  colorlinks=true,
  linkcolor={blue},
  filecolor={Maroon},
  citecolor={Blue},
  urlcolor={Blue},
  pdfcreator={LaTeX via pandoc}}

\title{Analyse statistique entre sexes}
\author{Aveneau / lange}
\date{}

\begin{document}
\maketitle
\ifdefined\Shaded\renewenvironment{Shaded}{\begin{tcolorbox}[borderline west={3pt}{0pt}{shadecolor}, sharp corners, interior hidden, enhanced, frame hidden, breakable, boxrule=0pt]}{\end{tcolorbox}}\fi

\hypertarget{description-du-projet}{%
\subsection{Description du projet}\label{description-du-projet}}

L'idée est de comparer s'il existe des différences de phénotypes
cliniques entre hommes et femmes dans la MCL.

\hypertarget{charger-le-jeu-de-donnuxe9es}{%
\subsection{Charger le jeu de
données}\label{charger-le-jeu-de-donnuxe9es}}

\begin{Shaded}
\begin{Highlighting}[]
\NormalTok{mcl }\OtherTok{\textless{}{-}} \FunctionTok{read.csv2}\NormalTok{(}\StringTok{"data/mcl.csv"}\NormalTok{)}
\end{Highlighting}
\end{Shaded}

Créer les paramètres nécessaires à l'analyse des données

\begin{Shaded}
\begin{Highlighting}[]
\NormalTok{mcl}\SpecialCharTok{$}\NormalTok{sexe }\OtherTok{\textless{}{-}} \FunctionTok{factor}\NormalTok{(mcl}\SpecialCharTok{$}\NormalTok{sexe, }\AttributeTok{levels=}\FunctionTok{c}\NormalTok{(}\DecValTok{1}\NormalTok{,}\DecValTok{2}\NormalTok{), }\AttributeTok{labels=}\FunctionTok{c}\NormalTok{(}\StringTok{"Woman"}\NormalTok{, }\StringTok{"Man"}\NormalTok{))}
\NormalTok{mcl}\SpecialCharTok{$}\NormalTok{etude }\OtherTok{\textless{}{-}} \FunctionTok{factor}\NormalTok{ (mcl}\SpecialCharTok{$}\NormalTok{etude, }\AttributeTok{levels=}\FunctionTok{c}\NormalTok{(}\DecValTok{1}\SpecialCharTok{:}\DecValTok{4}\NormalTok{), }\AttributeTok{labels=}\FunctionTok{c}\NormalTok{(}\StringTok{"No qualification"}\NormalTok{, }\StringTok{"Lower secondary education"}\NormalTok{, }\StringTok{"Upper secondary education"}\NormalTok{, }\StringTok{"Higher education"}\NormalTok{))}
\NormalTok{mcl}\SpecialCharTok{$}\NormalTok{tabac }\OtherTok{\textless{}{-}} \FunctionTok{factor}\NormalTok{ (mcl}\SpecialCharTok{$}\NormalTok{tabac, }\AttributeTok{levels=}\FunctionTok{c}\NormalTok{(}\DecValTok{0}\NormalTok{,}\DecValTok{1}\NormalTok{), }\AttributeTok{labels=}\FunctionTok{c}\NormalTok{(}\StringTok{"No"}\NormalTok{, }\StringTok{"Yes"}\NormalTok{))}
\NormalTok{mcl}\SpecialCharTok{$}\NormalTok{hta }\OtherTok{\textless{}{-}} \FunctionTok{factor}\NormalTok{ (mcl}\SpecialCharTok{$}\NormalTok{hta, }\AttributeTok{levels=}\FunctionTok{c}\NormalTok{(}\DecValTok{0}\NormalTok{,}\DecValTok{1}\NormalTok{), }\AttributeTok{labels=}\FunctionTok{c}\NormalTok{(}\StringTok{"No"}\NormalTok{, }\StringTok{"Yes"}\NormalTok{))}
\NormalTok{mcl}\SpecialCharTok{$}\NormalTok{diabete }\OtherTok{\textless{}{-}} \FunctionTok{factor}\NormalTok{ (mcl}\SpecialCharTok{$}\NormalTok{diabete, }\AttributeTok{levels=}\FunctionTok{c}\NormalTok{(}\DecValTok{0}\NormalTok{,}\DecValTok{1}\NormalTok{), }\AttributeTok{labels=}\FunctionTok{c}\NormalTok{(}\StringTok{"No"}\NormalTok{, }\StringTok{"Yes"}\NormalTok{))}
\NormalTok{mcl}\SpecialCharTok{$}\NormalTok{dyslipidemie }\OtherTok{\textless{}{-}} \FunctionTok{factor}\NormalTok{ (mcl}\SpecialCharTok{$}\NormalTok{dyslipidemie, }\AttributeTok{levels=}\FunctionTok{c}\NormalTok{(}\DecValTok{0}\NormalTok{,}\DecValTok{1}\NormalTok{), }\AttributeTok{labels=}\FunctionTok{c}\NormalTok{(}\StringTok{"No"}\NormalTok{, }\StringTok{"Yes"}\NormalTok{))}
\NormalTok{mcl}\SpecialCharTok{$}\NormalTok{saos }\OtherTok{\textless{}{-}} \FunctionTok{factor}\NormalTok{ (mcl}\SpecialCharTok{$}\NormalTok{saos, }\AttributeTok{levels=}\FunctionTok{c}\NormalTok{(}\DecValTok{0}\NormalTok{,}\DecValTok{1}\NormalTok{), }\AttributeTok{labels=}\FunctionTok{c}\NormalTok{(}\StringTok{"No"}\NormalTok{, }\StringTok{"Yes"}\NormalTok{))}
\NormalTok{mcl}\SpecialCharTok{$}\NormalTok{mabio }\OtherTok{\textless{}{-}} \FunctionTok{factor}\NormalTok{ (mcl}\SpecialCharTok{$}\NormalTok{mabio, }\AttributeTok{levels=}\FunctionTok{c}\NormalTok{(}\DecValTok{0}\NormalTok{,}\DecValTok{1}\NormalTok{), }\AttributeTok{labels=}\FunctionTok{c}\NormalTok{(}\StringTok{"No"}\NormalTok{, }\StringTok{"Yes"}\NormalTok{))}
\NormalTok{mcl}\SpecialCharTok{$}\NormalTok{rbd }\OtherTok{\textless{}{-}} \FunctionTok{factor}\NormalTok{ (mcl}\SpecialCharTok{$}\NormalTok{rbd, }\AttributeTok{levels=}\FunctionTok{c}\NormalTok{(}\DecValTok{0}\NormalTok{,}\DecValTok{1}\NormalTok{), }\AttributeTok{labels=}\FunctionTok{c}\NormalTok{(}\StringTok{"No"}\NormalTok{, }\StringTok{"Yes"}\NormalTok{))}
\NormalTok{mcl}\SpecialCharTok{$}\NormalTok{park }\OtherTok{\textless{}{-}} \FunctionTok{factor}\NormalTok{ (mcl}\SpecialCharTok{$}\NormalTok{park, }\AttributeTok{levels=}\FunctionTok{c}\NormalTok{(}\DecValTok{0}\NormalTok{,}\DecValTok{1}\NormalTok{), }\AttributeTok{labels=}\FunctionTok{c}\NormalTok{(}\StringTok{"No"}\NormalTok{, }\StringTok{"Yes"}\NormalTok{))}
\end{Highlighting}
\end{Shaded}

\hypertarget{test-univariuxe9-entre-sexe-et-troubles-du-sommeil}{%
\subsection{Test univarié entre sexe et troubles du
sommeil}\label{test-univariuxe9-entre-sexe-et-troubles-du-sommeil}}

\begin{Shaded}
\begin{Highlighting}[]
\NormalTok{mod }\OtherTok{\textless{}{-}} \FunctionTok{glm}\NormalTok{ (rbd }\SpecialCharTok{\textasciitilde{}}\NormalTok{ sexe, }\AttributeTok{data =}\NormalTok{ mcl, }\AttributeTok{family =} \StringTok{"binomial"}\NormalTok{)}
\FunctionTok{summary}\NormalTok{ (mod)}
\end{Highlighting}
\end{Shaded}

\begin{verbatim}

Call:
glm(formula = rbd ~ sexe, family = "binomial", data = mcl)

Coefficients:
            Estimate Std. Error z value Pr(>|z|)   
(Intercept)   0.2364     0.2442   0.968  0.33310   
sexeMan       0.9305     0.3153   2.952  0.00316 **
---
Signif. codes:  0 '***' 0.001 '**' 0.01 '*' 0.05 '.' 0.1 ' ' 1

(Dispersion parameter for binomial family taken to be 1)

    Null deviance: 254.40  on 206  degrees of freedom
Residual deviance: 245.69  on 205  degrees of freedom
  (103 observations effacées parce que manquantes)
AIC: 249.69

Number of Fisher Scoring iterations: 4
\end{verbatim}

\begin{Shaded}
\begin{Highlighting}[]
\FunctionTok{exp}\NormalTok{(}\FunctionTok{coefficients}\NormalTok{(mod))}
\end{Highlighting}
\end{Shaded}

\begin{verbatim}
(Intercept)     sexeMan 
   1.266667    2.535885 
\end{verbatim}

\hypertarget{segmentation-en-fonction-du-statut-mabio}{%
\subsection{Segmentation en fonction du statut
mabio}\label{segmentation-en-fonction-du-statut-mabio}}

Utilisation de la fonction \textbf{subset}

\begin{Shaded}
\begin{Highlighting}[]
\NormalTok{mcl\_ma }\OtherTok{\textless{}{-}} \FunctionTok{subset}\NormalTok{(mcl, mabio}\SpecialCharTok{==}\StringTok{"Yes"}\NormalTok{)}
\NormalTok{mcl\_nonma }\OtherTok{\textless{}{-}} \FunctionTok{subset}\NormalTok{(mcl, mabio}\SpecialCharTok{==}\StringTok{"No"}\NormalTok{)}
\end{Highlighting}
\end{Shaded}

\hypertarget{calcul-de-moduxe8les-univariuxe9s}{%
\subsection{Calcul de modèles
univariés}\label{calcul-de-moduxe8les-univariuxe9s}}

\begin{Shaded}
\begin{Highlighting}[]
\CommentTok{\#Pour ceux MA+}
\NormalTok{mod\_ma }\OtherTok{\textless{}{-}} \FunctionTok{glm}\NormalTok{ (rbd }\SpecialCharTok{\textasciitilde{}}\NormalTok{ sexe, }\AttributeTok{data =}\NormalTok{ mcl\_ma, }\AttributeTok{family =} \StringTok{"binomial"}\NormalTok{)}
\FunctionTok{summary}\NormalTok{ (mod\_ma)}
\end{Highlighting}
\end{Shaded}

\begin{verbatim}

Call:
glm(formula = rbd ~ sexe, family = "binomial", data = mcl_ma)

Coefficients:
            Estimate Std. Error z value Pr(>|z|)
(Intercept)   0.2231     0.6708   0.333    0.739
sexeMan       0.4700     1.0954   0.429    0.668

(Dispersion parameter for binomial family taken to be 1)

    Null deviance: 20.190  on 14  degrees of freedom
Residual deviance: 20.003  on 13  degrees of freedom
  (8 observations effacées parce que manquantes)
AIC: 24.003

Number of Fisher Scoring iterations: 4
\end{verbatim}

\begin{Shaded}
\begin{Highlighting}[]
\FunctionTok{exp}\NormalTok{(}\FunctionTok{coefficients}\NormalTok{(mod\_ma))}
\end{Highlighting}
\end{Shaded}

\begin{verbatim}
(Intercept)     sexeMan 
       1.25        1.60 
\end{verbatim}

\begin{Shaded}
\begin{Highlighting}[]
\CommentTok{\#Pour ceux MA{-}}
\NormalTok{mod\_nonma }\OtherTok{\textless{}{-}} \FunctionTok{glm}\NormalTok{ (rbd }\SpecialCharTok{\textasciitilde{}}\NormalTok{ sexe, }\AttributeTok{data =}\NormalTok{ mcl\_nonma, }\AttributeTok{family =} \StringTok{"binomial"}\NormalTok{)}
\FunctionTok{summary}\NormalTok{ (mod\_nonma)}
\end{Highlighting}
\end{Shaded}

\begin{verbatim}

Call:
glm(formula = rbd ~ sexe, family = "binomial", data = mcl_nonma)

Coefficients:
            Estimate Std. Error z value Pr(>|z|)  
(Intercept)   0.1054     0.4595   0.229   0.8186  
sexeMan       0.9933     0.5601   1.773   0.0762 .
---
Signif. codes:  0 '***' 0.001 '**' 0.01 '*' 0.05 '.' 0.1 ' ' 1

(Dispersion parameter for binomial family taken to be 1)

    Null deviance: 87.896  on 70  degrees of freedom
Residual deviance: 84.770  on 69  degrees of freedom
  (32 observations effacées parce que manquantes)
AIC: 88.77

Number of Fisher Scoring iterations: 4
\end{verbatim}

\begin{Shaded}
\begin{Highlighting}[]
\FunctionTok{exp}\NormalTok{(}\FunctionTok{coefficients}\NormalTok{(mod\_nonma))}
\end{Highlighting}
\end{Shaded}

\begin{verbatim}
(Intercept)     sexeMan 
   1.111111    2.700000 
\end{verbatim}

\hypertarget{conclusion}{%
\subsection{Conclusion}\label{conclusion}}

En gros, ici, la présence de biomarqueurs MA est un facteur
d'intéraction négatif sur la relation sexe / rbd. L'absence de
biomarqueurs MA amplifie l'association entre le sexe masculin et la
présence de rbd.

Pourquoi est-ce qu'on perd en significativité ?

Car il y a une perte d'effectif dans l'analyse :

\begin{Shaded}
\begin{Highlighting}[]
\FunctionTok{table}\NormalTok{(mcl}\SpecialCharTok{$}\NormalTok{sexe, mcl}\SpecialCharTok{$}\NormalTok{mabio, }\AttributeTok{useNA =} \StringTok{"always"}\NormalTok{)}
\end{Highlighting}
\end{Shaded}

\begin{verbatim}
       
         No Yes <NA>
  Woman  23  11   66
  Man    80  12  118
  <NA>    0   0    0
\end{verbatim}

L'analyse sans prise en compte des biomarqueurs le fait sur les 310
patients, celle en adaptant sur les biomarqueurs sur beaucoup moins de
patients.



\end{document}
